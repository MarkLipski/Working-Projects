\documentclass[conference]{IEEEtran}
\usepackage[margin=1in]{geometry}
\usepackage{cite}
\usepackage{amsmath,amssymb,amsfonts}
\usepackage{algorithmic}
\usepackage{graphicx}
\usepackage{textcomp}
\usepackage{xcolor}
\usepackage{subcaption}
\usepackage{cuted}

\title{A Four Terminal Charge Pump Model Incorporating Device Leakage Effects}
\author{Mark Lipski}

\begin{document}
	\maketitle
	\section{Abstract}
	This paper constructs a steady state model of the continuous conversion ratio charge pump architecture from first principles, incorporating incomplete charge transfer and finite switch resistance. The model is then verified using a pspice model of the circuit, and the limitations of the assumptions are discussed.
	\section{Introduction}
	
	
	\section{Steady State Model}
	The Four terminal charge pump device used for the model can be seen in Fig. \ref{}. The motivation for selecting the associated model is its usefulness across various charge pump based dc-dc architectures. The model can be used in the Dickson, Series Parallel, and ladder configurations, as well as any combination of the three. 
	
	Next, the framework for constructing the four terminal model is largely based upon the work in \cite{}, which incorporates the transistor resistance into the calculation for input and output current. 
	
	The leakage effects associated with the transistors are modelled using theory from the \cite{} transistor models. The testing is performed in pspice, using the PTM to simulate the transistors used. 
	
	
 	
 	\section{Conclusion}
 	In this paper, a steady state model of the continuous ratio charge pump architecture is presented. The model is then verified using pspice simulations, and the results are discussed.
 	
\end{document}