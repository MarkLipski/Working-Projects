\documentclass[conference]{IEEEtran}
\usepackage[margin=1in]{geometry}
\usepackage{cite}
\usepackage{amsmath,amssymb,amsfonts}
\usepackage{algorithmic}
\usepackage{graphicx}
\usepackage{textcomp}
\usepackage{xcolor}
\usepackage{subcaption}
\usepackage{cuted}

\title{A Four Terminal Charge Pump Model Incorporating Device Leakage Effects}
\author{Mark Lipski}

\begin{document}
	\maketitle
	\section{Abstract}
	This paper constructs a steady state model of the continuous conversion ratio charge pump architecture from first principles, incorporating incomplete charge transfer and finite switch resistance. The model is then verified using a pspice model of the circuit, and the limitations of the assumptions are discussed.
	\section{Introduction}
	
	
	\section{Steady State Model}
	The Four terminal charge pump device used for the model can be seen in Fig. \ref{}. The motivation for selecting the associated model is its usefulness across various charge pump based dc-dc architectures. The model can be used in the Dickson, Series Parallel, and ladder configurations, as well as any combination of the three. 
	
	The internal circuit representation of the four terminal model can be seen in Fig. \ref{}, where $R_{Fly} = \frac{1}{f_{SW}C_{Fly}}$. However, this model neglects the impact of incomplete charge transfer on the output charge delivered, which limits its usefulness.
	
	The effect of incomplete charge transfer on the output and input charge characteristics of a dc-dc converter has been studied \ref{}. However, the model studied assumed ideal clock generation. The following analysis incorporates the resistance of clock drivers into the equations for input and output current. 
	
	\subsection{Incomplete Charge Transfer}
	The associated analysis occurs at steady state  some assumptions:
	\begin{itemize}
		\item All the capacitors and resistors used in the analysis are linear and time invariant.
		\item The value of $\alpha_T + \alpha_B < 0.1$.
	\end{itemize}
	Consider Fig. \ref{}, this circuit has two equivalent circuit states, which can be represented as in Fig. \ref{} and \ref{}. The first order approximation of the voltage at $V_{TP1}$ is,
	\begin{equation}
	V_{TP1}(t) = \exp\left(\frac{-t}{\tau}\right)V_{TP1}^{i} + \left(1 - \exp\left(\frac{-t}{\tau}\right)\right)V_{T1},
	\end{equation}
	where $\tau$ is,
	\begin{equation}
	\tau = 2R_{ON}C_{Fly}\left(1+\frac{\alpha_B+\alpha_T}{2}\right),
	\end{equation}
	and $V_{TP1}^i$ is the initial value of $V_{TP1}$ at the start of the first phase. The final value of $V_{TP1}$ can be evaluated at $t = \frac{T_{SW}}{2}$, as phase 1 occupies half the period,
	\begin{equation}
	V_{TP1}^f = V_{TP1}\left(\frac{T_{SW}}{2}\right).
	\end{equation}	
	Finally, a substitution can be made, relating to the exponential decay at the end of the time step,
	\begin{equation}
	A = \exp\left(-\frac{T_{SW}}{2\tau}\right),
	\end{equation}
	which can be used to create a simplified expression for $V_{TP1}^f$,
	\begin{equation}
	V_{TP1}^f = AV_{TP1}^i + (1-A)V_{T1}.
	\end{equation}
	To solve for $V_{TP1}^i$, the initial condition on $C_{Fly}$ must be known, which can be acquired from the prior state of the capacitor, in this case $V_{TP2}^f - V_{BP2}^f$. This, can then be represented as a voltage source as in Fig. \ref{}, yielding an initial condition of,
	\begin{equation}
	V_{TP1}^i = \frac{V_{TP2}^f - V_{BP2}^f + V_{T1} + V_{B1}}{2}.
	\end{equation}
	This can then be substituted into the prior equation for $V_{TP1}^f$, resulting in
	\begin{equation}
	V_{TP1}^f = A\left(\tfrac{V_{TP2}^f - V_{BP2}^f + V_{T1} - V_{B1}}{2}\right) + (1-A)V_{T1}.
	\end{equation}
	A similar procedure can then be repeated for the other nodal voltages,
	\begin{equation}
	V_{BP1}^f = AV_{TP1}^i + (1-A)V_{B1},
	\end{equation}
	\begin{equation}
	V_{TP2}^f = AV_{TP2}^i + (1-A)V_{T2},
	\end{equation}
	\begin{equation}
	V_{BP2}^f = AV_{BP2}^i + (1-A)V_{B2}.
	\end{equation}
	The associated initial conditions for the variables are,
	\begin{equation}
	V_{BP1}^i = \tfrac{V_{BP2}^f - V_{TP2}^f + V_{B1} + V_{T1}}{2},
	\end{equation}
	\begin{equation}
	V_{TP2}^i = \tfrac{V_{TP1}^f - V_{BP1}^f + V_{T2} + V_{B2}}{2},
	\end{equation}
	\begin{equation}
	V_{BP2}^i = \tfrac{V_{BP1}^f - V_{TP1}^f + V_{T2} + V_{B2}}{2}.
	\end{equation}
	Using the equations for the initial conditions and the final voltages, the final value of $V_{TP1}$ can be solved,
	\begin{equation}
	V_{TP1}^f = V_{T1} + \Delta V, 
	\end{equation}
	\textbf{CHECK THIS}
	where 
	\begin{equation}
	\Delta V = \frac{A(V_{B1} - V_{T1} - V_{B2} + V_{T2})}{2(A+1)}.
	\end{equation}
	Similarly, the expressions for the other final voltages are,
	\begin{equation}
	V_{BP1}^f = V_{B1} - \Delta V,
	\end{equation}
	\begin{equation}
	V_{TP2}^f = V_{T2} - \Delta V,
	\end{equation}
	and
	\begin{equation}
	V_{BP2}^f = V_{B2} + \Delta V.
	\end{equation}
	This can then be used to generate the circuit model seen in Fig. \ref{}, which will incorporate the effects of incomplete charge transfer.
	
	
	
	The leakage effects associated with the transistors are modelled using theory from the \cite{} transistor models. The testing is performed in pspice, using the PTM to simulate the transistors used. 
	
	
\end{document}